\thispagestyle{empty}

\begin{Huge}
\textbf{Avant-propos}
\end{Huge}
\bigskip

S'il est une chose que l'on apprends très souvent d'expérience lors d'un premier emploi en développement logiciel, c'est l'importance, au sein d'une même équipe, d'un référentiel commun de règles qui régissent le développement. L'absence de telles règles étant, malheureusement souvent, à l'origine d'une mauvaise compréhension du code écrit et de sa difficulté de lecture. Hors, la première documentation que l'on puisse fournir à propos d'un code au sein d'une équipe de développement, c'est le code lui-même.\bigskip

Si certaines formations universitaires en informatique et génie logiciel incluent les outils et règles, tant implicites qu'explicites, dans leur programmes, il est plus rare qu'un étudiant d'une formation en électronique acquière au plus tôt ces notions dans son cycle d'apprentissage. Pour autant, dans le cas particulier d'électroniciens, il est de nos jours absolument inconcevable de ne pas maitriser des rudiment de programmation, ne serais-ce que pour pouvoir utiliser et designer correctement des systèmes incluant des micro-contrôleurs ou des circuits logiques programmables.\bigskip

Dans la mesure où je n'ai moi-même appris et compris l'importance de ces notions qu'une fois arrivé dans un milieu professionnel, j'ai voulu rédiger ce document que je souhaite le plus simple et clair possible afin de proposer un ensemble de règles de programmation commun à une équipe d'étudiants développant un programme. Ce document porte plus particulièrement sur des règles de rédaction du code en \textbf{Langage C} dans le contexte d'un système embarqué, mais peuvent être adaptées à d'autres langages et environnements de développement.\bigskip

\begin{flushright}
Vincent RICCHI
\end{flushright}

\pagebreak