\section{Variables}

\begin{center}
\begin{tabular}{|c p{12.3cm}|}
\hline
\rowcolor{red!10}\textbf{Règle 6.1} & On utilisera systématiquement les types de variables permettant d'identifier leur taille et le fait qu'elles soient signées ou non. \\ \hline
 & Ces types de variables sont définis par le compilateur comme étant : \\
 & uint8\_ t, uint16\_ t, uint32\_ t pour les types non signés \\
 & int8\_ t, int16\_ t, int32\_ t pour les type signés \\ \hline
\hline
\end{tabular}
\end{center}

\medskip

\begin{center}
\begin{tabular}{|c p{12.3cm}|}
\hline
\rowcolor{red!10}\textbf{Règle 6.2} & Toutes les variables, tableaux, structures et paramètres de sortie doivent êtres initialisées avant utilisation. \\ \hline
 & Si l’on appelle la valeur d’une variable alors que celle-ci n’a jamais été affectée, cela peut mener à un comportement indéfini du programme.\\ \hline
\hline
\end{tabular}
\end{center}

\medskip

\begin{center}
\begin{tabular}{|c p{12.3cm}|}
\hline
\rowcolor{red!10}\textbf{Règle 6.3} & Lors de l'initialisation par une valeur numérique d'une variable de type long le suffixe L doit être ajouté. \\ \hline
 & Dans la mesure où il n’est pas nécessaire de donner le format hexadécimal exact d’une variable de type long lors de son initialisation ou que celle-ci peut être faite avec une valeur décimale, l’utilisation du suffixe L permet de s’assurer que la variable est correctement initialisée.\\ \hline
\hline
\end{tabular}
\end{center}

\medskip

\begin{center}
\begin{tabular}{|c p{12.3cm}|}
\hline
\rowcolor{red!10}\textbf{Règle 6.4} & Lors de l'initialisation par une valeur numérique d'une variable de type non signé le suffixe U doit être ajouté.\\ \hline
 & Dans le cadre de l’initialisation d’une variable d’un type non signé en hexadécimal, il est préférable de faire appel au suffixe U pour clarifier la valeur à laquelle on fait appel.\\ \hline
\hline
\end{tabular}
\end{center}

\medskip

\pagebreak