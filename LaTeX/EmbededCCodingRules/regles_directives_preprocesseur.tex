\section{Directives préprocesseur}

\begin{center}
\begin{tabular}{|c l|}
\hline
\rowcolor{red!10}\textbf{Règle 4.1} & Les {\fontfamily{AnonymousPro}\selectfont\color{orange}{\# undef}} sont interdits. \\ \hline
 & Une directive {\fontfamily{AnonymousPro}\selectfont\color{orange}{\# undef}} peut rendre obscur le fait qu'une macro soit définie ou non. \\
 & Il est tout aussi efficace et plus clair de mettre en commentaire une \\
 & directive \# define afin de ne plus définir une macro utilisateur à la \\
 & compilation. De plus lors d'une recherche dans un fichier ou sur un ensemble \\
 & de fichiers, il est plus évident visuellement de comprendre si une macro est \\
 & définie ou non lorsque l'on rencontre {\fontfamily{AnonymousPro}\selectfont\color{green}{//\# define}} plutôt que {\fontfamily{AnonymousPro}\selectfont\color{orange}{\# undef}}. \\ \hline
\hline
\end{tabular}
\end{center}

\smallskip
\begin{large}
\textbf{\textsc{Exception :}}
\end{large}
Si la désactivation de mots clef du compilateur est nécessaire, on pourra utiliser la directive {\fontfamily{AnonymousPro}\selectfont\color{orange}{\# undef}}. Il faudra alors documenter et justifier son usage.

\medskip

\begin{center}
\begin{tabular}{|c l|}
\hline
\rowcolor{red!10}\textbf{Règle 4.2} & Une macro ne doit jamais pouvoir être confondue avec une constante. \\ \hline
 & Ce type de confusion à la lecture du code est aisément évitable et \\
 & simplifiera toujours la compréhension par un autre programmeur amené à \\
 & reprendre la partie du code en question. \\ \hline
\hline
\end{tabular}
\end{center}
 
\smallskip
\begin{large}
\textbf{\textsc{Exemple :}}
\end{large}
\lstinputlisting{macro_definition.c}

\pagebreak

\begin{center}
\begin{tabular}{|c l|}
\hline
\rowcolor{red!10}\textbf{Règle 4.3} & On préfèrera toujours écrire une fonction plutôt qu'une macro si les deux \\
\rowcolor{red!10} & sont interchangeables. \\ \hline
 & Une fonction est déboguable plus facilement qu'une macro. \\ \hline
\hline
\end{tabular}
\end{center}
 
\smallskip
\begin{large}
\textbf{\textsc{Exemple :}}
\end{large}
\lstinputlisting{macro_against_function.c}

\medskip

\begin{center}
\begin{tabular}{|c l|}
\hline
\rowcolor{red!10}\textbf{Règle 4.4} & Toute directive {\fontfamily{AnonymousPro}\selectfont\color{orange}{\# include}\color{mymauve}{"unFichier.h"}}/{\fontfamily{AnonymousPro}\selectfont\color{mymauve}{<unFichier.h>}} ne doit être précédées que \\
\rowcolor{red!10} & par des directives préprocesseur ou des commentaires. On placera \\
\rowcolor{red!10} & systématiquement les directives {\fontfamily{AnonymousPro}\selectfont\color{orange}{\# include}} en tête de fichier. \\ \hline
 & Placer les directives {\fontfamily{AnonymousPro}\selectfont\color{orange}{\# include}} en début de fichier permet de s'assurer \\
 & rapidement que les références nécessaires au fonctionnement du module sont \\
 & présentes. Si elles se situent de façon dispersée au travers du fichier, cela \\
 & peut entrainer un comportement indéfini si l'inclusion de certains modules \\
 & font appel à des références incluses seulement par d'autres plus bas dans le fichier \\ \hline
\hline
\end{tabular}
\end{center}
 
\smallskip
\begin{large}
\textbf{\textsc{Exemple :}}
\end{large}
Ce qui suit est une mauvaise pratique.

\lstinputlisting{include_location.c}

\medskip

\begin{center}
\begin{tabular}{|c l|}
\hline
\rowcolor{red!10}\textbf{Règle 4.5} & Les séquences de drapeaux de compilation {\fontfamily{AnonymousPro}\selectfont\color{orange}{\# if} 0 [...] \color{orange}{\# else} [...] \color{orange}{\# endif}}  \\
\rowcolor{red!10} & et {\fontfamily{AnonymousPro}\selectfont\color{orange}{\# if} 0 [...] \color{orange}{\# endif}} sont interdites. \\ \hline
 & L'utilisation de tels drapeaux de compilation rends immédiatement mort le \\
 & code encadré par {\fontfamily{AnonymousPro}\selectfont\color{orange}{\# if} 0 [...] \color{orange}{\# else}} ou par {\fontfamily{AnonymousPro}\selectfont\color{orange}{\# if} 0 [...] \color{orange}{\# endif}} endif. Selon \\
 & l'IDE utilisé, cela n'apparaitra pas de façon claire et évidente, pouvant \\
 & induire un programmeur en erreur. Cela peut tout aussi bien se faire en \\
 & passant le code concerné en commentaire. \\ \hline
\hline
\end{tabular}
\end{center}
 
\pagebreak
