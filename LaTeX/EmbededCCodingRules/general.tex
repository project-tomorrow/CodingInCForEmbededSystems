\documentclass[a4paper, titlepage]{article}

\usepackage{libertine}
\renewcommand*\familydefault{\sfdefault}  % Only if the base font of the document is to be sans serif
\usepackage{AnonymousPro}
\usepackage[T1]{fontenc}      % Police contenant les caractères français
\usepackage[utf8]{inputenc}   % LaTeX, comprends les accents !
\usepackage{geometry}         % Définir les marges
\usepackage{amssymb}
%NE PAS OUBLIER DE DECOMMENTER LES LIGNES SUIVANTES LORS DE L'INCLUSION DES IMAGES
\usepackage{fancyhdr}
\usepackage{graphicx}
\usepackage{color}
\usepackage[table]{xcolor}
%\usepackage{tikz}
\usepackage{bbding}
\usepackage{titlesec}
\usepackage[normalem]{ulem}
\usepackage[francais]{babel}  % Placez ici une liste de langues, la
                              % dernière étant la langue principale
\usepackage{listings}        % Supporte l'inclusion d'extraits de code avec leur mise en  forme

%\pagestyle{headings}        % Pour mettre des entètes avec les titres
                            % des sections en haut de page

% ---------------------------------------------
% - Definition du style des inclusion de code -
% ---------------------------------------------
\definecolor{mygreen}{rgb}{0,0.6,0}
%\definecolor{mygray}{rgb}{0.4,0.4,0.4}
\definecolor{mymauve}{rgb}{0.58,0,0.82}
\definecolor{lightgray}{rgb}{0.98,0.98,0.98}
\definecolor{orange}{rgb}{0.5,0.5,0}

\lstset{ %
  backgroundcolor=\color{lightgray},   % choose the background color; you must add \usepackage{color} or \usepackage{xcolor}; should come as last argument
  basicstyle=\fontfamily{AnonymousPro}\selectfont,            % the size of the fonts that are used for the code
  breakatwhitespace=false,         % sets if automatic breaks should only happen at whitespace
  breaklines=true,                 % sets automatic line breaking
  captionpos=false,                % sets the caption-position to bottom
  commentstyle=\slshape\color{mygreen},    % comment style
  deletekeywords={...},            % if you want to delete keywords from the given language
  escapeinside={\%*}{*)},          % if you want to add LaTeX within your code
  extendedchars=true,              % lets you use non-ASCII characters; for 8-bits encodings only, does not work with UTF-8
  frame=single,	                   % adds a frame around the code
  frameround = TTTT,               % rounds the frame corners
  keepspaces=true,                 % keeps spaces in text, useful for keeping indentation of code (possibly needs columns=flexible)
  keywordstyle=\bfseries\color{red},       % keyword style
  language=C,                      % the language of the code
  morekeywords={*,...},            % if you want to add more keywords to the set
  emph={define,ifdef,endif},    % add a empahsis on other keywords
  emphstyle=\color{orange},        % color on emphasised words
  emph={[2]typedef,struct,enum,unsigned,const,volatile
  ,void,char,int,long,float,uint8_t,uint16_t,uint32_t},    % add a empahsis on other keywords
  emphstyle={[2]\bfseries\color{blue}},       % color on emphasised words
  numbers=left,                    % where to put the line-numbers; possible values are (none, left, right)
  numbersep=10pt,                   % how far the line-numbers are from the code
  %numberstyle=\color{mygray}, % the style that is used for the line-numbers
  rulecolor=\color{black},         % if not set, the frame-color may be changed on line-breaks within not-black text (e.g. comments (green here))
  showspaces=false,                % show spaces everywhere adding particular underscores; it overrides 'showstringspaces'
  showstringspaces=false,          % underline spaces within strings only
  showtabs=false,                  % show tabs within strings adding particular underscores
  stepnumber=5,                    % the step between two line-numbers. If it's 1, each line will be numbered
  firstnumber=1,                  % number at which the code block starts
  stringstyle=\color{mymauve},     % string literal style
  tabsize=4,	                   % sets default tabsize to 2 spaces
  title=\lstname                   % show the filename of files included with \lstinputlisting; also try caption instead of title
}

% --------------------------------------------
% -------- Définition des en-tête ------------
% --------------------------------------------
\pagestyle{fancy}
\fancyhead{} % clear all header fields
\fancyhead[L]{\slshape \leftmark}
\fancyhead[R]{\includegraphics[scale=0.15]{ProToLogo.png}}
\fancyfoot{} % clear all footer fields
\fancyfoot[C]{\thepage}
\fancyfoot[L]{Association \textbf{ProTo}}
\renewcommand{\headrulewidth}{0.4pt}
\renewcommand{\footrulewidth}{0.4pt}

% ----------------------------------------------------------------------
% -------- Définition des format des titres de partie et de section ----
% ----------------------------------------------------------------------
\titleformat{\part}[frame]
  {\large\scshape}
  {\filright\enspace Partie \thepart\enspace}
  {8pt}
  {\Large\scshape\filcenter}

\titleformat{\section}[block]
  {\Large\bfseries\filright}
  {\fbox{\itshape\thesection}}
  {0.5em}{}

\begin{document}

% --------------------------------------------
% -------- Définition de la page de garde ----
% --------------------------------------------
\begin{titlepage}
  % \fontfamily{phv}\selectfont
  \vspace*{\stretch{1}}
  \begin{flushleft}\huge\bfseries\scshape
    Règles de Développement pour la Programmation en \textit{Langage C} dans le Cadre de Systèmes Embarqués
  \end{flushleft}
  \hrule
  \begin{flushright}\LARGE\slshape
    Embeded C Development - Coding Rules
  \end{flushright}
  \vspace*{\stretch{2}}
  \begin{center}
    Vincent \textsc{Ricchi} - Association \textbf{ProTo}\linebreak
    2017
  \end{center}
\end{titlepage}

% -------------------------------------------------------------------
% -------- Inclusion des différents fichiesr contenants le texte ----
% -------------------------------------------------------------------

\part*{Avant-propos}
\thispagestyle{empty}

\begin{Huge}
\textbf{Avant-propos}
\end{Huge}
\bigskip

S'il est une chose que l'on apprends très souvent d'expérience lors d'un premier emploi en développement logiciel, c'est l'importance, au sein d'une même équipe, d'un référentiel commun de règles qui régissent le développement. L'absence de telles règles étant, malheureusement souvent, à l'origine d'une mauvaise compréhension du code écrit et de sa difficulté de lecture. Hors, la première documentation que l'on puisse fournir à propos d'un code au sein d'une équipe de développement, c'est le code lui-même.\bigskip

Si certaines formations universitaires en informatique et génie logiciel incluent les outils et règles, tant implicites qu'explicites, dans leur programmes, il est plus rare qu'un étudiant d'une formation en électronique acquière au plus tôt ces notions dans son cycle d'apprentissage. Pour autant, dans le cas particulier d'électroniciens, il est de nos jours absolument inconcevable de ne pas maitriser des rudiment de programmation, ne serais-ce que pour pouvoir utiliser et designer correctement des systèmes incluant des micro-contrôleurs ou des circuits logiques programmables.\bigskip

Dans la mesure où je n'ai moi-même appris et compris l'importance de ces notions qu'une fois arrivé dans un milieu professionnel, j'ai voulu rédiger ce document que je souhaite le plus simple et clair possible afin de proposer un ensemble de règles de programmation commun à une équipe d'étudiants développant un programme. Ce document porte plus particulièrement sur des règles de rédaction du code en \textbf{Langage C} dans le contexte d'un système embarqué, mais peuvent être adaptées à d'autres langages et environnements de développement.\bigskip

\begin{flushright}
Vincent RICCHI
\end{flushright}

\pagebreak

\tableofcontents

\pagebreak

\part{Introduction}
\section{Objet du document}

Les règles ont pour but de garantir la lisibilité et l'inter-compatibilité du code écrit par chaque programmeur. Elles ont pour but d'être simples et de compréhension rapide.\bigskip

\section{Documents de référence}

\begin{itemize}
\item[\PencilRightDown] MISRA C:2012 Guidelines for the use of the C language in critical systems --- Mars 2013
\item[\PencilRightDown] MISRA C:2012 Amendment 1 Additional security guidelines for MISRA C:2012 --- Avril 2016
\item[\PencilRightDown] MISRA-C:2004 Guidelines for the use of the C language in critical systems --- Octobre 2004
\item[\PencilRightDown] MISRA C:2012 Addendum 1 Rule Mappings --- Mars 2013
\item[\PencilRightDown] ISO/IEC TS 17961 - C Secure Coding Rules --- 26 Juin 2012
\item[\PencilRightDown] MISRA C:2012 Addendum 2 Coverage of MISRA C:2012 against ISO/IEC TS 17961:2013 "C Secure" --- Avril 2016
\item[\PencilRightDown] ISO/IEC 9899:201x Programming languages - C --- 12 Avril 2011
\item[\PencilRightDown] Embeded C Coding Standard -- Barr Group --- 2013
\item[\PencilRightDown] C Coding Standard -- SEI CERT --- 2016 Edition
\item[\PencilRightDown] The Power of 10:
Rules for Developing Safety-Critical Code -- Gerard J. Holzmann from NASA/JPL Laboratory for Reliable Software --- Juin 2006
\item[\PencilRightDown] EPITA Coding Style Standard --- 9 Septembre 2003
\end{itemize}
\bigskip

\begin{large}
\textbf{Note}
\end{large}
\medskip

Ces documents ont servi de base pour écrire le document présent. Il ne s'agit pas ici d'intégrer un corpus de règles complet (tel que \textit{MISRA-C:2004} par exemple) dans la mesure où cela peut s'avérer trop contraignant ou même inadapté à notre environnement et contreproductif en fin de compte.
Nous avons choisi d'en reprendre certaines ou leur esprit, dans la mesure où elles nous semblaient simple de compréhension et adaptées à notre contexte de développement.

\pagebreak

\part{Présentation globale du document}
Ce document adopte les conventions de présentation de suivantes. \medskip

\section{Présentation du texte}

\begin{itemize}
\item Le texte général adopte comme unique langue la \textbf{langue française} à l’exception que quelques mots issus du jargon technique qui devront être définis dans le glossaire (par exemple le terme \textit{warning}).
\item Le texte général est écrit en police \xout{Times New Roman}\footnote{La police de ce document est encore indéterminée} et est justifié.
\item Un mot clef défini dans le glossaire \textit{apparait en italique}.
\item La mise en valeur d’un terme est faite par sa \textbf{mise en gras} dans le corps de texte
\item Il n’y a pas de cumul de mise en forme. Par exemple : \textbf{\textit{ceci a été banni}}.
\item Il n’y a pas de texte \textsl{souligné}.
\end{itemize}
\bigskip

\section{Présentation d'extraits de code}

\begin{itemize}
\item Le code adopte comme unique langue la langue \textbf{anglaise}.
\item Une citation de code dans un bloc de texte apparait en police \xout{Consolas}\footnote{La police des extraits de code de ce document est encore indéterminée} avec les options de mise en forme définies ci-dessous.
\item Un bloc de code apparait comme il suit :
\end{itemize}

% \begin{verbatim}
% En police Consolas dans un encadré séparé, sur fond légèrement grisé, avec une ligne
% vierge en fin de bloc et avec la mise en forme suivante :
% - Les types de variables et leurs mots clef associés (char, void, const,
% enum, typedef, etc…) sont en bleu et gras.
% - Les mots clef du langage C (if, while, sizeof, break, etc…) sont en rouge
% et gras
% - Les chaines de caractères et les caractères ("hello" et ‘\n’ par
% exemple) sont en violet
% - Les commentaires (/*…*/ ou //…) sont en vert
% - Les opérateurs (=, *, &, |, etc…) sont en gris
% - Les nombres (0x41, 1, …) sont en cyan
% - Les directives préprocesseur (#define, #ifdef, etc…) sont en orange
%\end{verbatim}
\lstinputlisting{code_bloc_layout.c}
\bigskip

\section{Présentation des règles}

\begin{itemize}
\item Les règles sont réparties en sous-groupes de thèmes communs (syntaxe, variable, etc…).
\item Chaque règle est numérotée \textbf{Règle X.Y} où \textbf{X} est le numéro du sous-groupe auquel appartient la règle et \textbf{Y} le numéro attribué à la règle dans son sous-groupe.
\item Une règle adopte la présentation type suivante\footnote{Texte Latin extrait de l’\textit{Énéide}} :
\end{itemize}

\begin{verbatim}
Règle X.Y : Énoncé de la Règle ; Arma uirumque cano, Troiae qui primus ab oris Italiam, fato profugus,
Lauiniaque uenit litora, multum ille et terris iactatus et alto
Texte explicatif ; ui superum saeuae memorem Iunonis ob iram ; multa quoque et bello passus,
dum conderet urbem, inferretque deos Latio, genus unde Latinum1,
\end{verbatim}
\begin{large}
\textbf{Exemple :}
\end{large}
\medskip
\lstinputlisting{eneide.c}

\bigskip

\pagebreak

\part{Règles détaillées}
\section{Compilation}

\begin{center}
\begin{tabular}{|c l|}
\hline
\rowcolor{red!10}\textbf{Règle 1.1} & Tous les fichiers sources doivent compiler sans warning. \\ \hline
 & 	Un warning témoigne d’un comportement risqué, non complètement défini ou sous optimal \\
 & du code compilé. Il se peut que cela ne pose aucun souci à l’instant où le code est écrit, \\
 & pour autant, cela ne veut pas dire qu’un comportement à risque ne sera pas déclenché par \\
 & du code futur faisant appel à la fonction concernée.\\ \hline
\hline
\end{tabular}
\end{center}

\medskip

\begin{center}
\begin{tabular}{|c l|}
\hline
\rowcolor{red!10}\textbf{Règle 1.2} & Le code est écrit et doit être compilé selon la norme \textit{C11}. \\ \hline
 & La norme \textit{C11} est à présent relativement ancienne, connue de la majorité des compilateurs \\
 & et robuste pour que l’on puisse se permettre d’en faire la norme standard dans laquelle le \\
 & code doit être compilé. Se limiter à des normes plus anciennes par habitude peut être \\
 & source de contraintes inutiles.\\ \hline
\hline
\end{tabular}
\end{center}

\medskip

\begin{center}
\begin{tabular}{|c l|}
\hline
\rowcolor{red!10}\textbf{Règle 1.3} & Il faut périodiquement lancer la toolchain cible pour s’assurer de \\
\rowcolor{red!10} & la compatibilité avec l’environnement cible. \\ \hline
 &  Si le code est développé sur hôte/PC avec gcc par exemple, il est nécessaire de \\
 & périodiquement lancer la toolchain cible, avec les adaptations nécessaires éventuelles, \\
 & pour s’assurer de la compatibilité du code développé avec l’environnement cible \\
 & auquel le binaire généré est destiné. \\ \hline
\hline
\end{tabular}
\end{center}

\medskip

\begin{center}
\begin{tabular}{|c l|}
\hline
\rowcolor{red!10}\textbf{Règle 1.4} & L’utilisation de balises préprocesseurs \# pragma changeant les \\
\rowcolor{red!10} & options de compilation devra être systématiquement documentée et justifiée. \\ \hline
 & L’utilisation de balises préprocesseur \# pragma permet de changer \\
 & les options de compilation au sein d’un fichier. Cela peut potentiellement \\
 & rompre l’homogénéité des règles de compilation appliquées au projet et peut \\
 & changer les propriété de la section de code compilé (taille du code compilé \\
 & plus grande car moins optimisé, temps d’exécution différent, etc…). \\ \hline
\hline
\end{tabular}
\end{center}

\pagebreak
\section{Philosophie}

\begin{center}
\begin{tabular}{|c l|}
\hline
\rowcolor{red!10}\textbf{Règle 2.1} & Garder à tout instant à l'esprit la philosophie \og KISS  : Keep It Smart and Simple \fg{}. \\
\rowcolor{red!10} & \quad trad. \og  Garde Cela Simple et Malin \fg{} \\ \hline
 & Garder le code simple, fonctionnel et si possible élégant. Il est important que d'autres \\
 & développeur du groupe puisse facilement comprendre le rôle d'une section de code et les \\
 & actions qu'il effectue. Dans cette optique, il est également important que le code ne \\
 & soit pas plus complexe que nécessaire (à ce titre voir \textbf{Règle 2.2}). \\ \hline
\hline
\end{tabular}
\end{center}

\smallskip
\begin{large}
\textbf{\textsc{Remarque :}}
\end{large}
De façon plus particulière, il ne faut jamais supposer que le code, aussi clairement rédigé soit-il, documente de façon suffisante par lui même le comportement attendu à l'exécution. Dans cette optique, il est d'importance capitale de commenter et de documenter toute partie du code qui ne serait pas triviale pour un néophyte. (voir \textbf{Figure \ref{self-documented-code} Page \pageref{self-documented-code}} pour une illustration humoristique)
\begin{figure}\centering
\includegraphics[scale=0.6]{pictures/CommitStrip_self-documented.jpg}
\caption{Code auto-documenté}
\label{self-documented-code}
\end{figure}

\medskip

\begin{center}
\begin{tabular}{|c l|}
\hline
\rowcolor{red!10}\textbf{Règle 2.2} & \og You Ain't Gonna Need It \fg{} est également un principe de base. \\
\rowcolor{red!10} & \quad trad. \og  Tu n'en Auras Pas Besoin \fg{} \\ \hline
 & Il est nécessaire de ne coder que ce qui répond à un besoin à un instant donné. Tout \\
 & code rédigé « en prévision de » constitue une perte de temps dans la réalisation de la \\
 & tache en cours et ne rendra l'ensemble du code écrit que plus complexe à valider et \\
 & seras potentiellement source de bugs. \\ \hline
\hline
\end{tabular}
\end{center}

\medskip

\begin{center}
\begin{tabular}{|c l|}
\hline
\rowcolor{red!10}\textbf{Règle 2.3} & Il ne doit pas y avoir de code inatteignable au sein d'un même projet. \\ \hline
 & Une section de code inatteignable est une section de code inutile ou désuète dans le \\
 & contexte où elle est identifiée. Conserver une section de code inatteignable alourdi \\
 & inutilement le code. \\ \hline
\hline
\end{tabular}
\end{center}

\smallskip
\begin{large}
\textbf{\textsc{Exception :}}
\end{large}
On pourra utiliser des drapeaux de compilation afin de gérer l'activation de fonctionnalités dans des sections de codes communes à plusieurs projets. Cela doit être documenté clairement.

\medskip

\begin{center}
\begin{tabular}{|c l|}
\hline
\rowcolor{red!10}\textbf{Règle 2.4} & Le code mis en commentaire est interdit. \\ \hline
 & Le code mis en commentaire constitue une pollution visuelle pour le programmeur et peut \\
 & potentiellement l'induire en erreur en lui proposant d'effectuer une action qui peut \\
 & s'avérer au mieux non nécessaire et au pire potentiellement dangereuse. \\ \hline
\hline
\end{tabular}
\end{center}

\smallskip
\begin{large}
\textbf{\textsc{Exception :}}
\end{large}
Dans un bloc de commentaire descriptif, on pourra faire figurer du code d'exemple, clairement documenté comme tel.


\pagebreak


\section{Organisation des fichiers sources}

\begin{center}
\begin{tabular}{|c l|}
\hline
\rowcolor{red!10}\textbf{Règle 3.1} & Toutes les précautions doivent être prises concernant l'inclusion des \textit{headers} afin \\
\rowcolor{red!10} & qu'ils ne soient pas inclus plus d'une fois dans un même objet. \\ \hline
 & Dans la mesure où de nombreux fichiers peuvent faire appel à un même module \\
 & il est important qu'on ne puisse pas faire d'inclusion récursive d'une même \\
 & source. On veillera donc à déclarer des drapeaux empêchant cela comme il suit. \\ \hline
\hline
\end{tabular}
\end{center}

\smallskip
\begin{large}
\textbf{\textsc{Exemple :}}
\end{large}
\lstinputlisting{header_file_inclusion.c}

\medskip

\begin{center}
\begin{tabular}{|c l|}
\hline
\rowcolor{red!10}\textbf{Règle 3.2} & Tout projet doit être organisé en modules logiciels. Un module logiciel est un \\
\rowcolor{red!10} & ensemble composé d'au moins un fichier \textbf{.h} et un fichier \textbf{.c}. \\ \hline
 & S'il ne doit y en avoir qu'un, le fichier \textbf{.h} ou \textit{header}, représente l'interface, \\
 & partie visible du module pour les autres modules du projet. \\
 & Il contient les prototypes des fonctions et les déclarations de variables et \\
 & constantes accessibles aux autres modules. Un fichier d'interface adoptera un \\
 & nom du type {\fontfamily{AnonymousPro}\selectfont my\textunderscore module\textunderscore file-itf.h}. \\
 & Le fichier \textbf{.c} contient l'implémentation. \\ \hline
\hline
\end{tabular}
\end{center}

\medskip

\begin{center}
\begin{tabular}{|c l|}
\hline
\rowcolor{red!10}\textbf{Règle 3.3} & Les noms des fichiers sources doivent être explicites et faire référence à \\
\rowcolor{red!10} & l'élément du programme qu'ils concernent.\\ \hline
 & Des noms de fichiers explicitent permettent de naviguer plus facilement dans \\
 & un projet et évitent d'avoir à fournir un effort de mémorisation des différents \\
 & rôles attribués à des noms de fichiers obscurs.\\ \hline
\hline
\end{tabular}
\end{center}

\smallskip
\begin{large}
\textbf{\textsc{Exemple :}}
\end{large}
Les fichiers sources ayant trait à la gestion des protocoles de communication sur un port série se nomment {\fontfamily{AnonymousPro}\selectfont serial\textunderscore [nom\textunderscore du\textunderscore module].[extension]}, tels que {\fontfamily{AnonymousPro}\selectfont serial\textunderscore interruptions.c} ou {\fontfamily{AnonymousPro}\selectfont serial\textunderscore port\textunderscore names.h}.

\medskip

\begin{center}
\begin{tabular}{|c l|}
\hline
\rowcolor{red!10}\textbf{Règle 3.4} & Le contenu d'un fichier \textbf{.h} ou \textbf{.c} doit être cohérent d'un point de vue fonctionnel.\\
 & Si plusieurs implémentations se chevauchent dans un même fichier, alors \\
 & celui-ci devra être séparé en plusieurs fichier cohérent d'un point de vue \\
 & fonctionnel. Le découpage du code en plusieurs parties permet d'en améliorer \\
 & la lisibilité. \\ \hline
\hline
\end{tabular}
\end{center}

\medskip

\begin{center}
\begin{tabular}{|c l|}
\hline
\rowcolor{red!10}\textbf{Règle 3.5} & Il est fortement souhaité que chaque fichier \textbf{.h} et \textbf{.c} contienne un cartouche descriptif \\
\rowcolor{red!10} & du rôle du module. \\ \hline
 & Au sein de \textsc{ProTo} nous utiliserons le cartouche ci-dessous. \\
 & Les balises \textcolor{purple}{@} permettent d'identifier les balises qui génère la documentation \\
 & avec un programme tel que \textit{Doxygen}. \\ \hline
\hline
\end{tabular}
\end{center}

\smallskip
\lstinputlisting{generic_file_header.c}

\smallskip
\begin{large}
\textbf{\textsc{Remarque :}}
\end{large}
Il n'est pas nécessaire d'indiquer une valeur à la balise \textcolor{purple}{@file}.

\medskip

\begin{center}
\begin{tabular}{|c l|}
\hline
\rowcolor{red!10}\textbf{Règle 3.6} & Les fonctions doivent posséder en en-tête un cartouche descriptif de leur rôle, leurs \\
\rowcolor{red!10} & paramètres d'entrées-sorties et de retour ainsi qu'un bref descriptif de leur \\
\rowcolor{red!10} & fonctionnement. \\ \hline
 & Cela permet une compréhension plus simple du code par l'ensemble des développeurs. \\
 & Au sein de \textsc{ProTo} nous utiliserons un cartouche similaire à celui présenté en exemple \\
 & ci-dessous. \\
 & On le placera dans le fichier \textbf{.c}. \\
 & On ajoutera les champs {\fontfamily{AnonymousPro}\selectfont [in]} et {\fontfamily{AnonymousPro}\selectfont [out]} dans le descriptif des paramètre afin de \\
 & préciser si les valeurs passées sont des valeurs d'entrée ou de sortie de \\
 & la fonction. \\
 & Les balises \textcolor{purple}{@} permettent d'identifier les balises qui génèrent la documentation avec un \\
 & programme tel que \textit{Doxygen}. \\ \hline
\hline
\end{tabular}
\end{center}

\smallskip
\begin{large}
\textbf{\textsc{Exemple :}}
\end{large}
Le cartouche est similaire à ce qui suit
\lstinputlisting{function_description_header.c}

\pagebreak
\section{Directives préprocesseur}

\begin{center}
\begin{tabular}{|c p{12.3cm}|}
\hline
\rowcolor{red!10}\textbf{Règle 4.1} & Les {\fontfamily{AnonymousPro}\selectfont\color{orange}{\#{}undef}} sont interdits. \\ \hline
 & Une directive {\fontfamily{AnonymousPro}\selectfont\color{orange}{\# undef}} peut rendre obscur le fait qu'une macro soit définie ou non. \\
 & Il est tout aussi efficace et plus clair de mettre en commentaire une directive \#{}define afin de ne plus définir une macro utilisateur à la compilation. De plus lors d'une recherche dans un fichier ou sur un ensemble de fichiers, il est plus évident visuellement de comprendre si une macro est définie ou non lorsque l'on rencontre {\fontfamily{AnonymousPro}\selectfont\color{green}{//\# define}} plutôt que {\fontfamily{AnonymousPro}\selectfont\color{orange}{\# undef}}. \\ \hline
\hline
\end{tabular}
\end{center}

\smallskip
\begin{large}
\textbf{\textsc{Exception :}}
\end{large}
Si la désactivation de mots clef du compilateur est nécessaire, on pourra utiliser la directive {\fontfamily{AnonymousPro}\selectfont\color{orange}{\# undef}}. Il faudra alors documenter et justifier son usage.

\medskip

\begin{center}
\begin{tabular}{|c p{12.3cm}|}
\hline
\rowcolor{red!10}\textbf{Règle 4.2} & Une macro ne doit jamais pouvoir être confondue avec une constante. \\ \hline
 & Ce type de confusion à la lecture du code est aisément évitable et simplifiera toujours la compréhension par un autre programmeur amené à reprendre la partie du code en question. \\ \hline
\hline
\end{tabular}
\end{center}
 
\smallskip
\begin{large}
\textbf{\textsc{Exemple :}}
\end{large}
\lstinputlisting{code_c_samples/macro_definition.c}

\pagebreak

\begin{center}
\begin{tabular}{|c p{12.3cm}|}
\hline
\rowcolor{red!10}\textbf{Règle 4.3} & On préfèrera toujours écrire une fonction plutôt qu'une macro si les deux sont interchangeables. \\ \hline
 & Une fonction est déboguable plus facilement qu'une macro. \\ \hline
\hline
\end{tabular}
\end{center}
 
\smallskip
\begin{large}
\textbf{\textsc{Exemple :}}
\end{large}
\lstinputlisting{code_c_samples/macro_against_function.c}

\medskip

\begin{center}
\begin{tabular}{|c p{12.3cm}|}
\hline
\rowcolor{red!10}\textbf{Règle 4.4} & Toute directive {\fontfamily{AnonymousPro}\selectfont\color{orange}{\# include}\color{mymauve}{"unFichier.h"}}/{\fontfamily{AnonymousPro}\selectfont\color{mymauve}{<unFichier.h>}} ne doit être précédées que par des directives préprocesseur ou des commentaires. On placera systématiquement les directives {\fontfamily{AnonymousPro}\selectfont\color{orange}{\# include}} en tête de fichier. \\ \hline
 & Placer les directives {\fontfamily{AnonymousPro}\selectfont\color{orange}{\# include}} en début de fichier permet de s'assurer rapidement que les références nécessaires au fonctionnement du module sont présentes. Si elles se situent de façon dispersée au travers du fichier, cela peut entrainer un comportement indéfini si l'inclusion de certains modules font appel à des références incluses seulement par d'autres plus bas dans le fichier \\ \hline
\hline
\end{tabular}
\end{center}
 
\smallskip
\begin{large}
\textbf{\textsc{Exemple :}}
\end{large}
Ce qui suit est une mauvaise pratique.

\lstinputlisting{code_c_samples/include_location.c}

\medskip

\begin{center}
\begin{tabular}{|c p{12.3cm}|}
\hline
\rowcolor{red!10}\textbf{Règle 4.5} & Les séquences de drapeaux de compilation {\fontfamily{AnonymousPro}\selectfont\color{orange}{\#{}if} 0 [...] \color{orange}{\#{}else} [...] \color{orange}{\#{}endif}} et {\fontfamily{AnonymousPro}\selectfont\color{orange}{\#{}if} 0 [...] \color{orange}{\#{}endif}} sont interdites. \\ \hline
 & L'utilisation de tels drapeaux de compilation rends immédiatement mort le code encadré par {\fontfamily{AnonymousPro}\selectfont\color{orange}{\#{}if} 0 [...] \color{orange}{\#{}else}} ou par {\fontfamily{AnonymousPro}\selectfont\color{orange}{\#{}if} 0 [...] \color{orange}{\#{}endif}}. Selon l'\textit{IDE} utilisé, cela n'apparaitra pas de façon claire et évidente, pouvant induire un programmeur en erreur. Cela peut tout aussi bien se faire en passant le code concerné en commentaire. \\ \hline
\hline
\end{tabular}
\end{center}
 
\pagebreak

\section{Mots clef du \textit{Langage C}}

\begin{center}
\begin{tabular}{|c l|}
\hline
\rowcolor{red!10}\textbf{Règle 5.1} & Les Digraphes et Trigraphes sont interdits. \\ \hline
 & Ils peuvent être remplacé par des symboles standards et ne font que rendre le \\
 & code moins clair et cassent potentiellement la compilation lorsque ceux-ci sont \\
 & utilisés par un programmeur qui ne les maitrise pas. De plus, ils ne sont pas \\
 & supportés par certains compilateurs. \\ \hline
\hline
\end{tabular}
\end{center}

\smallskip 

Les Digraphes sont les symboles suivants :
\smallskip 
\lstinputlisting{code_c_samples/digraph_symbol_correspondance.c}


Trigraphes sont les symboles suivants :
\smallskip 
\lstinputlisting{code_c_samples/trigraph_symbol_correspondance.c}

\medskip

\begin{center}
\begin{tabular}{|c l|}
\hline
\rowcolor{red!10}\textbf{Règle 5.2} & Les opérations ternaires (\textit{inline if} en anglais) sont interdites. \\ \hline
 & Les affectations de valeur ternaires rendent le code inutilement complexe dans \\
 & sa lecture.\\ \hline
\hline
\end{tabular}
\end{center}

\pagebreak

\begin{large}
\textbf{\textsc{Exemple :}}
\end{large}
\smallskip
\lstinputlisting{code_c_samples/operateur_ternaire.c}

\medskip

\begin{center}
\begin{tabular}{|c l|}
\hline
\rowcolor{red!10}\textbf{Règle 5.3} & Les fonctions récursives sont interdites. \\ \hline
 & Les fonctions récursives consomment potentiellement beaucoup de ressources en \\
 & RAM et dans un contexte de développement embarqué, cela peut mener à un \\
 & dépassement mémoire.\\ \hline
\hline
\end{tabular}
\end{center}

\medskip

\begin{center}
\begin{tabular}{|c l|}
\hline
\rowcolor{red!10}\textbf{Règle 5.4} & Le mot clef {\fontfamily{AnonymousPro}\selectfont\bfseries\color{red}{goto}} est interdit. \\ \hline
 & Le mot clef {\fontfamily{AnonymousPro}\selectfont\bfseries\color{red}{goto}} peut potentiellement induire un dysfonctionnement du programme \\
 & en sautant d'une section de code à l'autre dans laquelle certaines variables ou \\
 & pointeurs n'ont plus de valeur cohérente. \\ \hline
\hline
\end{tabular}
\end{center}

\smallskip 

\begin{large}
\textbf{\textsc{Exception :}}
\end{large}
Si l'appel à une instruction {\fontfamily{AnonymousPro}\selectfont\bfseries\color{red}{goto}} est vraiment nécessaire, elle doit se faire de manière descendante dans la fonction et le code doit être commenté de façon explicite.

\medskip

\begin{center}
\begin{tabular}{|c l|}
\hline
\rowcolor{red!10}\textbf{Règle 5.5} & Les fonctions ou objets ne doivent pas être définis plus d'une fois \\ \hline
 & Redéfinir une fonction ou un objet amène un fort risque de télescopage à la \\
 & compilation. \\ \hline
\hline
\end{tabular}
\end{center}

\medskip

\begin{center}
\begin{tabular}{|c l|}
\hline
\rowcolor{red!10}\textbf{Règle 5.6} & Toute occurrence d'une fonction pouvant avoir un comportement indéfini doit être \\
\rowcolor{red!10} & encadrée afin d'être contrôlée. \\ \hline
 & Tester les paramètres pour s'assurer qu'ils sont consistant avant l'appel à une \\
 & fonction pouvant avoir un comportement indéfini permet de fortement restreindre \\
 & le risque de voir celui-ci se produire et nous permet de nous placer dans une \\
 & optique de code robuste. \\ \hline
\hline
\end{tabular}
\end{center}

\medskip

\begin{center}
\begin{tabular}{|c l|}
\hline
\rowcolor{red!10}\textbf{Règle 5.7} & Il ne doit pas y avoir de conversion implicite. \\ \hline
 & Les conversions implicites provoquent des \textit{warnings}.\\ \hline
\hline
\end{tabular}
\end{center}

\medskip

\begin{center}
\begin{tabular}{|c l|}
\hline
\rowcolor{red!10}\textbf{Règle 5.8} & Il ne doit pas y avoir d'appel/définition de fonction implicite. \\ \hline
 & Les appels/définitions de fonctions implicites signifient qu'un module du \\
 & programme n'est pas visible depuis le module courant ou qu'une fonction n'est pas \\
 & définie correctement. Cela provoque l'apparition de \textit{warnings} \\ \hline
\hline
\end{tabular}
\end{center}

\medskip

\begin{center}
\begin{tabular}{|c l|}
\hline
\rowcolor{red!10}\textbf{Règle 5.9} & Les allocations dynamiques de mémoire sont interdite. \\ \hline
 & Les allocations dynamiques de mémoire sont des procédures risquées dans un cadre \\
 & embarqués et peuvent être éviter en dimensionnant correctement la taille des \\ 
 & piles allouées à chaque tache lors de la conception.\\ \hline
\hline
\end{tabular}
\end{center}

\medskip

\begin{center}
\begin{tabular}{|c l|}
\hline
\rowcolor{red!10}\textbf{Règle 5.10} & L'utilisation de fonction {\fontfamily{AnonymousPro}\selectfont\bfseries\color{red}{inline}} est interdite. \\ \hline
 & L'utilisation de fonctions {\fontfamily{AnonymousPro}\selectfont\bfseries\color{red}{inline}} duplique le code lors de la compilation et en \\
 & augmente la taille finale dans un contexte où il est nécessaire de faire des \\
 & économies de mémoire. \\ \hline
\hline
\end{tabular}
\end{center}

\pagebreak

\section{Variables}

\begin{center}
\begin{tabular}{|c l|}
\hline
\rowcolor{red!10}\textbf{Règle 6.1} & On utilisera systématiquement les types de variables permettant d'identifier \\
\rowcolor{red!10} & leur taille et le fait qu'elles soient signées ou non. \\ \hline
 & Ces types de variables sont définis par le compilateur comme étant : \\
 & uint8\_ t, uint16\_ t, uint32\_ t pour les types non signés \\
 & int8\_ t, int16\_ t, int32\_ t pour les type signés \\ \hline
\hline
\end{tabular}
\end{center}

\medskip

\begin{center}
\begin{tabular}{|c l|}
\hline
\rowcolor{red!10}\textbf{Règle 6.2} & Toutes les variables, tableaux, structures et paramètres de sortie doivent êtres\\
\rowcolor{red!10} & initialisées avant utilisation. \\ \hline
 & Si l’on appelle la valeur d’une variable alors que celle-ci n’a jamais été \\
 & affectée, cela peut mener à un comportement indéfini du programme.\\ \hline
\hline
\end{tabular}
\end{center}

\medskip

\begin{center}
\begin{tabular}{|c l|}
\hline
\rowcolor{red!10}\textbf{Règle 6.3} & Lors de l'initialisation par une valeur numérique d'une variable de type long\\
\rowcolor{red!10} & le suffixe L doit être ajouté. \\ \hline
 & Dans la mesure où il n’est pas nécessaire de donner le format hexadécimal exact \\
 & d’une variable de type long lors de son initialisation ou que celle-ci peut être\\
 & faite avec une valeur décimale, l’utilisation du suffixe L permet de s’assurer \\
 & que la variable est correctement initialisée.\\ \hline
\hline
\end{tabular}
\end{center}

\medskip

\begin{center}
\begin{tabular}{|c l|}
\hline
\rowcolor{red!10}\textbf{Règle 6.4} & Lors de l'initialisation par une valeur numérique d'une variable de type non \\
\rowcolor{red!10} & signé le suffixe U doit être ajouté.\\ \hline
 & Dans le cadre de l’initialisation d’une variable d’un type non signé en \\
 & hexadécimal, il est préférable de faire appel au suffixe U pour clarifier la \\
 & valeur à laquelle on fait appel.\\ \hline
\hline
\end{tabular}
\end{center}

\medskip

\pagebreak
\section{Style du code}

\begin{center}
\begin{tabular}{|c l|}
\hline
\rowcolor{red!10}\textbf{Règle 7.1} & La mise en page des instruction conditionnelles ou de boucles ne doit pas prêter\\
\rowcolor{red!10} & à confusion. À cette fin, elle devra suivre la syntaxe définie ci-dessous.\\ \hline
 & L’utilisation d’accolades ouvrantes et fermantes pour toutes les instructions \\
 & conditionnelles et les boucles permet de visualiser facilement le code qui \\
 & s’exécutera lors du passage dans la section du code concernée. Cela permet \\
 & d’éviter la mauvaise compréhension par de futurs nouveaux programmeurs de \\
 & l’équipe.\\ \hline
\hline
\end{tabular}
\end{center}

\smallskip
\begin{large}
\textbf{\textsc{Exemple :}}
\end{large}
Le code suivant respecte une mise en page correcte.
\smallskip
\lstinputlisting{code_c_samples/if_statement_shape.c}

\smallskip
\begin{large}
\textbf{\textsc{Exemple :}}
\end{large}
La mise en forme suivante est incorrecte.
\smallskip
\lstinputlisting{code_c_samples/incorect_if_statement.c}

\medskip

\begin{center}
\begin{tabular}{|c l|}
\hline
\rowcolor{red!10}\textbf{Règle 7.2a} & Tous les noms de variables commencent en minuscule.\\ \hline
 & C’est une convention de typage.\\ \hline
\hline
\end{tabular}
\end{center}

\smallskip
\begin{large}
\textbf{\textsc{Exemple :}}
\end{large}
\lstinputlisting[firstline=1, lastline=2]{code_c_samples/variable_macro_type_style_definition.c}

\medskip

\begin{center}
\begin{tabular}{|c l|}
\hline
\rowcolor{red!10}\textbf{Règle 7.2b} & Tous les noms de types commencent par une majuscule et se terminent par \_ t.\\ \hline
 & C’est une convention de typage.\\ \hline
\hline
\end{tabular}
\end{center}

\smallskip
\begin{large}
\textbf{\textsc{Exception :}}
\end{large}
Les types appartenant aux mots clefs du langage C ou à des bibliothèques de fonctions extérieures ne sont pas concernés.

\smallskip
\begin{large}
\textbf{\textsc{Exemple :}}
\end{large}
\lstinputlisting[firstline=3, lastline=4]{code_c_samples/variable_macro_type_style_definition.c}

\medskip

\begin{center}
\begin{tabular}{|c l|}
\hline
\rowcolor{red!10}\textbf{Règle 7.2c} & Tous les noms de macro et constantes définies par \# define sont en majuscule.\\ \hline
 & C’est une convention de typage.\\ \hline
\hline
\end{tabular}
\end{center}

\smallskip
\begin{large}
\textbf{\textsc{Exemple :}}
\end{large}
\lstinputlisting[firstline=5, lastline=6]{code_c_samples/variable_macro_type_style_definition.c}

\medskip

\begin{center}
\begin{tabular}{|c l|}
\hline
\rowcolor{red!10}\textbf{Règle 7.3a} & Il est préférable d’utiliser // ... pour les commentaires d’une ligne ou en \\
\rowcolor{red!10} & fin de ligne.\\ \hline
 & C’est un choix de convention d’écriture.\\ \hline
\hline
\end{tabular}
\end{center}

\medskip

\begin{center}
\begin{tabular}{|c l|}
\hline
\rowcolor{red!10}\textbf{Règle 7.3b} & Il est préférable d’utiliser /* ... */ pour les commentaires de plus d’une ligne.\\ \hline
 & C’est un choix de convention d’écriture. \\ \hline
\hline
\end{tabular}
\end{center}

Afin que cela soit plus lisible on écriera les blocs de commentaires comme dans l’exemple suivant.

\smallskip
\begin{large}
\textbf{\textsc{Exemple :}}
\end{large}
\lstinputlisting{code_c_samples/comment_block_style.c}

\smallskip
\begin{large}
\textbf{\textsc{Remarque :}}
\end{large}
L’utilisation de /** en début de bloc de commentaire permet d’identifier un bloc de commentaire qui sera traité par Doxygen ou tout autre logiciel similaire

\medskip

\begin{center}
\begin{tabular}{|c l|}
\hline
\rowcolor{red!10}\textbf{Règle 7.4} & L’indentation du code se fait au moyen de 4 caractères espaces.\\ \hline
 & C’est un choix de convention de présentation du code. \\ \hline
\hline
\end{tabular}
\end{center}

\medskip

\begin{center}
\begin{tabular}{|c l|}
\hline
\rowcolor{red!10}\textbf{Règle 7.5} & Les noms de fonctions publiques d’un modules devront s’écrire \\
\rowcolor{red!10} & [MODULE\_ TAG]\_ nomDeFonction(...). Les fonctions internes aux modules seront \\
\rowcolor{red!10} & laissées à la discrétion du programmeur.\\ \hline
 & C’est un choix de convention de présentation du code. \\ \hline
\hline
\end{tabular}
\end{center}

\pagebreak

\appendix

\pagebreak

\part*{Glossaire}
\begin{description}
\item[C99/C11] Jargon usuel désignant les normes ISO/IEC 9899:1999 et ISO/IEC 9899:201x définissant le standard du langage de programmation C.
\item[Binaire] Système de numération en base 2.\linebreak
Dans le jargon de programmeur, ce terme désigne également le format d’un fichier de programme, compilé depuis un langage de programmation (langage C dans notre cas) vers le langage de la machine cible sur laquelle le programme devra s’exécuter.
\item[Doxygen] Programme permettant la génération automatique de documentation du code à l’aide de balises.
\item[gcc] \textbf{G}NU \textbf{C} \textbf{C}ompiler : compilateur pour le langage C GNU.
\item[Header] Fichier d’en-tête du langage C et C++.
\item[IDE] \textbf{I}ntegrated \textbf{D}evelopment \textbf{E}nvironement : Environnement de développement.\linebreak
Ensemble d’outils informatique permettant le développement d’une solution logicielle (Keil $\mu$Vision, Eclipse, …) ou matérielle (Altium Designer, AutoCAD EAGLE, …).
\end{description}
\pagebreak

\pagebreak

\part*{Résumé des règles}
\begin{description}
\item[1] \textsc{Compilation}
\item[Règle 1.1] Tous les fichiers sources doivent compiler sans warning.
\item[Règle 1.2] Le code est écrit et doit être compilé selon la norme C99.
\item[Règle 1.3] Il faut périodiquement lancer la toolchain cible pour s’assurer de la compatibilité avec l’environnement cible.
\item[Règle 1.4] L’utilisation de directives préprocesseurs \# pragma changeant les options de compilation devra être systématiquement documentée et justifiée.

\item[2] \textsc{Philosophie du code}
\item[Règle 2.1] Garder à tout instant à l’esprit la philosophie \og KISS  : Keep It Smart and Simple \fg{}. \linebreak
trad. \og Garde Cela Simple et Malin \fg{}
\item[Règle 2.2] \og You Ain’t Gonna Need It \fg{} est également un principe de base. \linebreak
trad. \og Tu n’en Auras Pas Besoin \fg{}
\item[Règle 2.3] Il ne doit pas y avoir de code inatteignable au sein d’un même projet.
\item[Règle 2.4] Le code mis en commentaire est interdit.

\item[3] \textsc{Organisation des fichiers sources}
\item[Règle 3.1] Toutes les précautions doivent être prises concernant l'inclusion des headers afin qu'ils ne soient pas inclus plus d'une fois dans un même objet.
\item[Règle 3.2] Tout projet doit être organisé en modules logiciels. Un module logiciel est un ensemble composé d’au moins un fichier .h et un fichier .c.
\item[Règle 3.3] Les noms des fichiers sources doivent être explicites et faire référence à l’élément du programme qu’ils concernent.
\item[Règle 3.4] Le contenu d’un fichier .h ou .c doit être cohérent d’un point de vue fonctionnel.
\item[Règle 3.5] Il est fortement souhaité que chaque fichier .h et .c contienne un cartouche descriptif du rôle du module.
\item[Règle 3.6] Les fonctions doivent posséder en en-tête un cartouche descriptif de leur rôle, leurs paramètres d’entrées-sorties et de retour ainsi qu’un bref descriptif de leur fonctionnement.

\item[4] \textsc{Directives preprocesseurs}
\item[Règle 4.1] Les \# undef sont interdits.
\item[Règle 4.2] Une macro ne doit jamais pouvoir être confondue avec une constante.
\item[Règle 4.3] On préfèrera toujours écrire une fonction plutôt qu'une macro si les deux sont interchangeables.
\item[Règle 4.4] Toute directive \# include "unFichier.h"/<unFichier.h> ne doit être précédées que par des directives préprocesseur ou des commentaires. On placera systématiquement les directives \# include en tête de fichier.
\item[Règle 4.5] Les séquences de drapeaux de compilation \# if 0 [...] \# else [...] \# endif et \# if 0 [...] \# endif sont interdites.

\item[5] \textsc{Comportement fonctionnel et mots clef du langage C}
\item[Règle 5.1] Les Digraphes et Trigraphes sont interdits.
\item[Règle 5.2] Les opérations ternaires (inline if en anglais) sont interdites.
\item[Règle 5.3] Les fonctions récursives sont interdites.
\item[Règle 5.4] Le mot clef goto est interdit.
\item[Règle 5.5] Les fonctions ou objets ne doivent pas être définis plus d'une fois.
\item[Règle 5.6] Toute occurrence d'une fonction pouvant avoir un comportement indéfini doit être encadrée afin d'être contrôlée.
\item[Règle 5.7] Il ne doit pas y avoir de conversion implicite.
\item[Règle 5.8] Il ne doit pas y avoir d'appel/définition de fonction implicite.
\item[Règle 5.9] Il ne doit pas y avoir d'allocation dynamique de mémoire.
\item[Règle 5.10] L’utilisation de fonction inline est interdite.

\item[6] \textsc{Variables}
\item[Règle 6.1] On utilisera systématiquement les types de variables permettant d'identifier leur taille et le fait qu'elles soient signées ou non.
\item[Règle 6.2] Toutes les variables, tableaux, structures et paramètres de sortie doivent êtres initialisées avant utilisation.
\item[Règle 6.3] Lors de l'initialisation par une valeur numérique d'une variable de type long le suffixe L doit être ajouté.
\item[Règle 6.4] Lors de l'initialisation par une valeur numérique d'une variable de type non signé le suffixe U doit être ajouté.

\item[7] \textsc{Style du code}
\item[Règle 7.1] La mise en page des instruction conditionnelles ou de boucles ne doit pas prêter à confusion. À cette fin, elle devra suivre la syntaxe définie.
\item[Règle 7.2a] Tous les noms de variables commencent en minuscule. Ils sont écrits en style camelCase.
\item[Règle 7.2b] Tous les noms de types commencent par une majuscule et se terminent par \_ t. Ils sont écrits en style camelCase.
\item[Règle 7.2c] Tous les noms de macro et constantes définies par \# define sont en majuscule.
\item[Règle 7.3a] Il est préférable d’utiliser // … pour les commentaires d’une ligne ou en fin de ligne.
\item[Règle 7.3b] Il est préférable d’utiliser /* … */ pour les commentaires de plus d’une ligne.
\item[Règle 7.4] L’indentation du code se fait au moyen de tabulation.
\item[Règle 7.5] Les noms de fonctions publiques d’un modules devront s’écrire [MODULE\_ TAG]\_ [nomDeFonction](...). Les fonctions internes aux modules seront laissées à la discrétion du programmeur.

\end{description}

\pagebreak

\end{document}