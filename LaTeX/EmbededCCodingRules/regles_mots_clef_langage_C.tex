\section{Mots clef du \textit{Langage C}}

\begin{center}
\begin{tabular}{|c p{12.3cm}|}
\hline
\rowcolor{red!10}\textbf{Règle 5.1} & Les Digraphes et Trigraphes sont interdits. \\ \hline
 & Ils peuvent être remplacé par des symboles standards et ne font que rendre le code moins clair et cassent potentiellement la compilation lorsque ceux-ci sont utilisés par un programmeur qui ne les maitrise pas. De plus, ils ne sont pas supportés par certains compilateurs. \\ \hline
\hline
\end{tabular}
\end{center}

\smallskip 

Les Digraphes sont les symboles suivants :
\smallskip 
\lstinputlisting{code_c_samples/digraph_symbol_correspondance.c}


Trigraphes sont les symboles suivants :
\smallskip 
\lstinputlisting{code_c_samples/trigraph_symbol_correspondance.c}

\medskip

\begin{center}
\begin{tabular}{|c p{12.3cm}|}
\hline
\rowcolor{red!10}\textbf{Règle 5.2} & Les opérations ternaires (\textit{inline if} en anglais) sont interdites. \\ \hline
 & Les affectations de valeur ternaires rendent le code inutilement complexe dans sa lecture.\\ \hline
\hline
\end{tabular}
\end{center}

\pagebreak

\begin{large}
\textbf{\textsc{Exemple :}}
\end{large}
\smallskip
\lstinputlisting{code_c_samples/operateur_ternaire.c}

\medskip

\begin{center}
\begin{tabular}{|c p{12.3cm}|}
\hline
\rowcolor{red!10}\textbf{Règle 5.3} & Les fonctions récursives sont interdites. \\ \hline
 & Les fonctions récursives consomment potentiellement beaucoup de ressources en RAM et dans un contexte de développement embarqué, cela peut mener à un dépassement mémoire.\\ \hline
\hline
\end{tabular}
\end{center}

\medskip

\begin{center}
\begin{tabular}{|c p{12.3cm}|}
\hline
\rowcolor{red!10}\textbf{Règle 5.4} & Le mot clef {\fontfamily{AnonymousPro}\selectfont\bfseries\color{red}{goto}} est interdit. \\ \hline
 & Le mot clef {\fontfamily{AnonymousPro}\selectfont\bfseries\color{red}{goto}} peut potentiellement induire un dysfonctionnement du programme en sautant d'une section de code à l'autre dans laquelle certaines variables ou pointeurs n'ont plus de valeur cohérente. \\ \hline
\hline
\end{tabular}
\end{center}

\smallskip 

\begin{large}
\textbf{\textsc{Exception :}}
\end{large}
Si l'appel à une instruction {\fontfamily{AnonymousPro}\selectfont\bfseries\color{red}{goto}} est vraiment nécessaire, elle doit se faire de manière descendante dans la fonction et le code doit être commenté de façon explicite.

\medskip

\begin{center}
\begin{tabular}{|c p{12.3cm}|}
\hline
\rowcolor{red!10}\textbf{Règle 5.5} & Les fonctions ou objets ne doivent pas être définis plus d'une fois \\ \hline
 & Redéfinir une fonction ou un objet amène un fort risque de télescopage à la compilation. \\ \hline
\hline
\end{tabular}
\end{center}

\medskip

\begin{center}
\begin{tabular}{|c p{12.3cm}|}
\hline
\rowcolor{red!10}\textbf{Règle 5.6} & Toute occurrence d'une fonction pouvant avoir un comportement indéfini doit être encadrée afin d'être contrôlée. \\ \hline
 & Tester les paramètres pour s'assurer qu'ils sont consistant avant l'appel à une fonction pouvant avoir un comportement indéfini permet de fortement restreindre le risque de voir celui-ci se produire et nous permet de nous placer dans une optique de code robuste. \\ \hline
\hline
\end{tabular}
\end{center}

\medskip

\begin{center}
\begin{tabular}{|c p{12.3cm}|}
\hline
\rowcolor{red!10}\textbf{Règle 5.7} & Il ne doit pas y avoir de conversion implicite. \\ \hline
 & Les conversions implicites provoquent des \textit{warnings}.\\ \hline
\hline
\end{tabular}
\end{center}

\medskip

\begin{center}
\begin{tabular}{|c p{12.3cm}|}
\hline
\rowcolor{red!10}\textbf{Règle 5.8} & Il ne doit pas y avoir d'appel/définition de fonction implicite. \\ \hline
 & Les appels/définitions de fonctions implicites signifient qu'un module du programme n'est pas visible depuis le module courant ou qu'une fonction n'est pas définie correctement. Cela provoque l'apparition de \textit{warnings} \\ \hline
\hline
\end{tabular}
\end{center}

\medskip

\begin{center}
\begin{tabular}{|c p{12.3cm}|}
\hline
\rowcolor{red!10}\textbf{Règle 5.9} & Les allocations dynamiques de mémoire sont interdite. \\ \hline
 & Les allocations dynamiques de mémoire sont des procédures risquées dans un cadre embarqués et peuvent être éviter en dimensionnant correctement la taille des piles allouées à chaque tache lors de la conception.\\ \hline
\hline
\end{tabular}
\end{center}

\medskip

\begin{center}
\begin{tabular}{|c p{12.3cm}|}
\hline
\rowcolor{red!10}\textbf{Règle 5.10} & L'utilisation de fonction {\fontfamily{AnonymousPro}\selectfont\bfseries\color{red}{inline}} est interdite. \\ \hline
 & L'utilisation de fonctions {\fontfamily{AnonymousPro}\selectfont\bfseries\color{red}{inline}} duplique le code lors de la compilation et en augmente la taille finale dans un contexte où il est nécessaire de faire des économies de mémoire. \\ \hline
\hline
\end{tabular}
\end{center}

\pagebreak
