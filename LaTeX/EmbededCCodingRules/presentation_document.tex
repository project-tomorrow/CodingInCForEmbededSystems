Ce document adopte les conventions de présentation de suivantes. \medskip

\section{Présentation du texte}

\begin{itemize}
\item Le texte général adopte comme unique langue la \textbf{langue française} à l’exception que quelques mots issus du jargon technique qui devront être définis dans le glossaire (par exemple le terme \textit{warning}).
\item Le texte général est écrit en police \textbf{Biolinum} et est justifié.
\item Un mot clef défini dans le glossaire \textit{apparait en italique}.
\item La mise en valeur d’un terme est faite par sa \textbf{mise en gras} dans le corps de texte
\item Il n’y a pas de cumul de mise en forme. Par exemple : \textbf{\textit{ceci a été banni}}.
\item Il n’y a pas de texte \underbar{souligné}.
\end{itemize}
\bigskip

\section{Présentation d'extraits de code}

\begin{itemize}
\item Le code adopte comme unique langue la langue \textbf{anglaise}.
\item Une citation de code apparait dans un encadré séparé, en police {\fontfamily{AnonymousPro}\selectfont\textbf{AnonymousPro}} avec les options de mise en forme définies ci-dessous :
\end{itemize}

\lstinputlisting{code_layout.c}
% L'ancien texte se trouve dans le fichier suivant
% \lstinputlisting{code_bloc_layout.c}

\pagebreak

\section{Présentation des règles}

\begin{itemize}
\item Les règles sont réparties en groupes de thèmes communs (syntaxe, variable, etc…).
\item Chaque règle est numérotée \textbf{Règle X.Y} où \textbf{X} est le numéro du groupe auquel appartient la règle et \textbf{Y} le numéro attribué à la règle dans son sous-groupe.
\item Une règle adopte la présentation type suivante\footnote{Texte Latin extrait de l’\textit{Énéide}} :
\end{itemize}

\medskip

\begin{center}
\begin{tabular}{|c l|}
\hline
\rowcolor{red!10}\textbf{Règle X.Y} & Énoncé de la Règle ; \textit{Arma uirumque cano, Troiae qui primus ab oris Italiam,} \\
\rowcolor{red!10} & \textit{fato profugus, Lauiniaque uenit litora, multum ille et terris iactatus et alto} \\ \hline
 & Texte explicatif ; \textit{ui superum saeuae memorem Iunonis ob iram ; multa quoque et} \\
 & \textit{bello passus, dum conderet urbem, inferretque deos Latio, genus unde Latinum,} \\ \hline
\hline
\end{tabular}
\end{center}

\medskip

\begin{large}
\textbf{Exemple :}
\end{large}
\medskip
\lstinputlisting{eneide.c}

\bigskip

\pagebreak