% -----------------------------------------  
% Autogenerated LaTeX file from XML DocBook  
% -----------------------------------------  
%%<params>
%% document.language en
%%</params>
\documentclass{article}
\usepackage{ifthen}
\newboolean{DBKIsBook}
\setboolean{DBKIsBook}{false}
\IfFileExists{ifxetex.sty}{%
    \usepackage{ifxetex}%
  }{%
    \newif\ifxetex
    \xetexfalse
  }
  \ifxetex
\usepackage{fontspec}
\usepackage{xltxtra}
\defaultfontfeatures{Mapping=tex-text}
\setmainfont{DejaVu Serif}
\setsansfont{DejaVu Sans}
\setmonofont{DejaVu Sans Mono}
\else
\usepackage[T1]{fontenc}
\usepackage[latin1]{inputenc}
\fi
\usepackage{fancybox}
\usepackage{makeidx}
\def\hyperparam{colorlinks,linkcolor=blue,pdfstartview=FitH}
\def\DBKpublisher{\includegraphics{dblatex}}
\usepackage[hyperlink]{asciidoc-dblatex}
\renewcommand{\DBKreleaseinfo}{}
\setcounter{tocdepth}{5}
\setcounter{secnumdepth}{3}
\title{C Coding Rules}
\author{Author is Vincent Ricchi}
\hypersetup{%
pdfcreator={DBLaTeX-0.3.9-3},%
pdftitle={C Coding Rules},%
pdfauthor={Author is Vincent Ricchi}}
\renewcommand{\DBKindexation}{}
\makeindex
\makeglossary
\begin{document}
\lstsetup
\frontmatter
\maketitle
\tableofcontents
\mainmatter
\begin{center}
\begingroup%
\setlength{\newtblsparewidth}{\linewidth-2\tabcolsep-2\tabcolsep-2\tabcolsep}%
\setlength{\newtblstarfactor}{\newtblsparewidth / \real{100}}%
\begin{longtable}{ll}\caption[{Multiline cells, row/col span}]{Multiline cells, row/col span}\tabularnewline
\endfirsthead
\caption[]{(continued)}\tabularnewline
\endhead
\hline
\multicolumn{1}{|p{50\newtblstarfactor}|}{\raggedright%
\textbf{R�gle 1.1}
%
}&\multicolumn{1}{p{50\newtblstarfactor}|}{\raggedright%
Garder � tout instant � l'esprit la philosophie \emph{� KISS  : Keep It Smart and Simple �}.
%
}\tabularnewline
\cline{1-1}\cline{2-2}\multicolumn{1}{|p{50\newtblstarfactor}|}{\raggedright%
%
}&\multicolumn{1}{p{50\newtblstarfactor}|}{\raggedright%
Garder le code simple, fonctionnel et si possible �l�gant. Il est important que d'autres d�veloppeur du groupe puisse facilement comprendre le r�le d'une section de code et les actions qu'il effectue. Dans cette optique, il est �galement important que le code ne soit pas plus complexe que n�cessaire (� ce titre voir \textbf{R�gle 2.2}).
%
}\tabularnewline
\hline
\end{longtable}\endgroup%

\end{center}

\end{document}
