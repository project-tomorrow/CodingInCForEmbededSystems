\section{Objet du document}

Les règles ont pour but de garantir la lisibilité et l'inter-compatibilité du code écrit par chaque programmeur. Elles ont pour but d'être simples et de compréhension rapide.\bigskip

\section{Documents de référence}

\begin{itemize}
\item[\PencilRightDown] MISRA C:2012 Guidelines for the use of the C language in critical systems --- Mars 2013
\item[\PencilRightDown] MISRA C:2012 Amendment 1 Additional security guidelines for MISRA C:2012 --- Avril 2016
\item[\PencilRightDown] MISRA-C:2004 Guidelines for the use of the C language in critical systems --- Octobre 2004
\item[\PencilRightDown] MISRA C:2012 Addendum 1 Rule Mappings --- Mars 2013
\item[\PencilRightDown] ISO/IEC TS 17961 - C Secure Coding Rules --- 26 Juin 2012
\item[\PencilRightDown] MISRA C:2012 Addendum 2 Coverage of MISRA C:2012 against ISO/IEC TS 17961:2013 "C Secure" --- Avril 2016
\item[\PencilRightDown] ISO/IEC 9899:201x Programming languages - C --- 12 Avril 2011
\item[\PencilRightDown] Embeded C Coding Standard -- Barr Group --- 2013
\item[\PencilRightDown] C Coding Standard -- SEI CERT --- 2016 Edition
\item[\PencilRightDown] The Power of 10:
Rules for Developing Safety-Critical Code -- Gerard J. Holzmann from NASA/JPL Laboratory for Reliable Software --- Juin 2006
\item[\PencilRightDown] EPITA Coding Style Standard --- 9 Septembre 2003
\end{itemize}
\bigskip

\begin{large}
\textbf{Note}
\end{large}
\medskip

Ces documents ont servi de base pour écrire le document présent. Il ne s'agit pas ici d'intégrer un corpus de règles complet (tel que \textit{MISRA-C:2004} par exemple) dans la mesure où cela peut s'avérer trop contraignant ou même inadapté à notre environnement et contreproductif en fin de compte.
Nous avons choisi d'en reprendre certaines ou leur esprit, dans la mesure où elles nous semblaient simple de compréhension et adaptées à notre contexte de développement.

\pagebreak