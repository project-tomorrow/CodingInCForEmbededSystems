\section{Organisation des fichiers sources}

\begin{center}
\begin{tabular}{|c p{12.3cm}|}
\hline
\rowcolor{red!10}\textbf{Règle 3.1} & Toutes les précautions doivent être prises concernant l'inclusion des \textit{headers} afin qu'ils ne soient pas inclus plus d'une fois dans un même objet. \\ \hline
 & Dans la mesure où de nombreux fichiers peuvent faire appel à un même module il est important qu'on ne puisse pas faire d'inclusion récursive d'une même source. On veillera donc à déclarer des drapeaux empêchant cela comme il suit. \\ \hline
\hline
\end{tabular}
\end{center}

\smallskip
\begin{large}
\textbf{\textsc{Exemple :}}
\end{large}
\lstinputlisting{code_c_samples/header_file_inclusion.c}

\medskip

\begin{center}
\begin{tabular}{|c p{12.3cm}|}
\hline
\rowcolor{red!10}\textbf{Règle 3.2} & Tout projet doit être organisé en modules logiciels. Un module logiciel est un ensemble composé d'au moins un fichier \textbf{.h} et un fichier \textbf{.c}. \\ \hline
 & S'il ne doit y en avoir qu'un, le fichier \textbf{.h} ou \textit{header}, représente l'interface, partie visible du module pour les autres modules du projet. Il contient les prototypes des fonctions et les déclarations de variables et constantes accessibles aux autres modules. Un fichier d'interface adoptera un nom du type {\fontfamily{AnonymousPro}\selectfont my\textunderscore module\textunderscore file-itf.h}. \\
 & Le fichier \textbf{.c} contient l'implémentation. \\ \hline
\hline
\end{tabular}
\end{center}

\medskip

\begin{center}
\begin{tabular}{|c p{12.3cm}|}
\hline
\rowcolor{red!10}\textbf{Règle 3.3} & Les noms des fichiers sources doivent être explicites et faire référence à l'élément du programme qu'ils concernent.\\ \hline
 & Des noms de fichiers explicitent permettent de naviguer plus facilement dans un projet et évitent d'avoir à fournir un effort de mémorisation des différents rôles attribués à des noms de fichiers obscurs.\\ \hline
\hline
\end{tabular}
\end{center}

\smallskip
\begin{large}
\textbf{\textsc{Exemple :}}
\end{large}
Les fichiers sources ayant trait à la gestion des protocoles de communication sur un port série se nomment {\fontfamily{AnonymousPro}\selectfont serial\textunderscore [nom\textunderscore du\textunderscore module].[extension]}, tels que {\fontfamily{AnonymousPro}\selectfont serial\textunderscore interruptions.c} ou {\fontfamily{AnonymousPro}\selectfont serial\textunderscore port\textunderscore names.h}.

\medskip

\begin{center}
\begin{tabular}{|c p{12.3cm}|}
\hline
\rowcolor{red!10}\textbf{Règle 3.4} & Le contenu d'un fichier \textbf{.h} ou \textbf{.c} doit être cohérent d'un point de vue fonctionnel.\\
 & Si plusieurs implémentations se chevauchent dans un même fichier, alors celui-ci devra être séparé en plusieurs fichier cohérent d'un point de vue fonctionnel. Le découpage du code en plusieurs parties permet d'en améliorer la lisibilité. \\ \hline
\hline
\end{tabular}
\end{center}

\medskip

\begin{center}
\begin{tabular}{|c p{12.3cm}|}
\hline
\rowcolor{red!10}\textbf{Règle 3.5} & Il est fortement souhaité que chaque fichier \textbf{.h} et \textbf{.c} contienne un cartouche descriptif du rôle du module. \\ \hline
 & Au sein de \textsc{ProTo} nous utiliserons le cartouche ci-dessous. \\
 & Les balises \textcolor{purple}{@} permettent d'identifier les balises qui génère la documentation avec un programme tel que \textit{Doxygen}. \\ \hline
\hline
\end{tabular}
\end{center}

\smallskip
\lstinputlisting{code_c_samples/generic_file_header.c}

\smallskip
\begin{large}
\textbf{\textsc{Remarque :}}
\end{large}
Il n'est pas nécessaire d'indiquer une valeur à la balise \textcolor{purple}{@file}.

\medskip

\begin{center}
\begin{tabular}{|c p{12.3cm}|}
\hline
\rowcolor{red!10}\textbf{Règle 3.6} & Les fonctions doivent posséder en en-tête un cartouche descriptif de leur rôle, leurs paramètres d'entrées-sorties et de retour ainsi qu'un bref descriptif de leur fonctionnement. \\ \hline
 & Cela permet une compréhension plus simple du code par l'ensemble des développeurs. Au sein de \textsc{ProTo} nous utiliserons un cartouche similaire à celui présenté en exemple ci-dessous. \\
 & On le placera dans le fichier \textbf{.h}. \\
 & On ajoutera les champs {\fontfamily{AnonymousPro}\selectfont [in]} et {\fontfamily{AnonymousPro}\selectfont [out]} dans le descriptif des paramètre afin de préciser si les valeurs passées sont des valeurs d'entrée ou de sortie de la fonction. \\
 & Les balises \textcolor{purple}{@} permettent d'identifier les balises qui génèrent la documentation avec un programme tel que \textit{Doxygen}. \\ \hline
\hline
\end{tabular}
\end{center}

\smallskip
\begin{large}
\textbf{\textsc{Exemple :}}
\end{large}
Le cartouche est similaire à ce qui suit
\lstinputlisting{code_c_samples/function_description_header.c}

\pagebreak