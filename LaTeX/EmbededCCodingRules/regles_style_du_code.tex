\section{Style du code}

\begin{center}
\begin{tabular}{|c l|}
\hline
\rowcolor{red!10}\textbf{Règle 7.1} & La mise en page des instruction conditionnelles ou de boucles ne doit pas prêter\\
\rowcolor{red!10} & à confusion. À cette fin, elle devra suivre la syntaxe définie ci-dessous.\\ \hline
 & L’utilisation d’accolades ouvrantes et fermantes pour toutes les instructions \\
 & conditionnelles et les boucles permet de visualiser facilement le code qui \\
 & s’exécutera lors du passage dans la section du code concernée. Cela permet \\
 & d’éviter la mauvaise compréhension par de futurs nouveaux programmeurs de \\
 & l’équipe.\\ \hline
\hline
\end{tabular}
\end{center}

\smallskip
\begin{large}
\textbf{\textsc{Exemple :}}
\end{large}
Le code suivant respecte une mise en page correcte.
\smallskip
\lstinputlisting{code_c_samples/if_statement_shape.c}

\smallskip
\begin{large}
\textbf{\textsc{Exemple :}}
\end{large}
La mise en forme suivante est incorrecte.
\smallskip
\lstinputlisting{code_c_samples/incorect_if_statement.c}

\medskip

\begin{center}
\begin{tabular}{|c l|}
\hline
\rowcolor{red!10}\textbf{Règle 7.2a} & Tous les noms de variables commencent en minuscule.\\ \hline
 & C’est une convention de typage.\\ \hline
\hline
\end{tabular}
\end{center}

\smallskip
\begin{large}
\textbf{\textsc{Exemple :}}
\end{large}
\lstinputlisting[firstline=1, lastline=2]{code_c_samples/variable_macro_type_style_definition.c}

\medskip

\begin{center}
\begin{tabular}{|c l|}
\hline
\rowcolor{red!10}\textbf{Règle 7.2b} & Tous les noms de types commencent par une majuscule et se terminent par \_ t.\\ \hline
 & C’est une convention de typage.\\ \hline
\hline
\end{tabular}
\end{center}

\smallskip
\begin{large}
\textbf{\textsc{Exception :}}
\end{large}
Les types appartenant aux mots clefs du langage C ou à des bibliothèques de fonctions extérieures ne sont pas concernés.

\smallskip
\begin{large}
\textbf{\textsc{Exemple :}}
\end{large}
\lstinputlisting[firstline=3, lastline=4]{code_c_samples/variable_macro_type_style_definition.c}

\medskip

\begin{center}
\begin{tabular}{|c l|}
\hline
\rowcolor{red!10}\textbf{Règle 7.2c} & Tous les noms de macro et constantes définies par \# define sont en majuscule.\\ \hline
 & C’est une convention de typage.\\ \hline
\hline
\end{tabular}
\end{center}

\smallskip
\begin{large}
\textbf{\textsc{Exemple :}}
\end{large}
\lstinputlisting[firstline=5, lastline=6]{code_c_samples/variable_macro_type_style_definition.c}

\medskip

\begin{center}
\begin{tabular}{|c l|}
\hline
\rowcolor{red!10}\textbf{Règle 7.3a} & Il est préférable d’utiliser // ... pour les commentaires d’une ligne ou en \\
\rowcolor{red!10} & fin de ligne.\\ \hline
 & C’est un choix de convention d’écriture.\\ \hline
\hline
\end{tabular}
\end{center}

\medskip

\begin{center}
\begin{tabular}{|c l|}
\hline
\rowcolor{red!10}\textbf{Règle 7.3b} & Il est préférable d’utiliser /* ... */ pour les commentaires de plus d’une ligne.\\ \hline
 & C’est un choix de convention d’écriture. \\ \hline
\hline
\end{tabular}
\end{center}

Afin que cela soit plus lisible on écriera les blocs de commentaires comme dans l’exemple suivant.

\smallskip
\begin{large}
\textbf{\textsc{Exemple :}}
\end{large}
\lstinputlisting{code_c_samples/comment_block_style.c}

\smallskip
\begin{large}
\textbf{\textsc{Remarque :}}
\end{large}
L’utilisation de /** en début de bloc de commentaire permet d’identifier un bloc de commentaire qui sera traité par Doxygen ou tout autre logiciel similaire

\medskip

\begin{center}
\begin{tabular}{|c l|}
\hline
\rowcolor{red!10}\textbf{Règle 7.4} & L’indentation du code se fait au moyen de 4 caractères espaces.\\ \hline
 & C’est un choix de convention de présentation du code. \\ \hline
\hline
\end{tabular}
\end{center}

\medskip

\begin{center}
\begin{tabular}{|c l|}
\hline
\rowcolor{red!10}\textbf{Règle 7.5} & Les noms de fonctions publiques d’un modules devront s’écrire \\
\rowcolor{red!10} & [MODULE\_ TAG]\_ nomDeFonction(...). Les fonctions internes aux modules seront \\
\rowcolor{red!10} & laissées à la discrétion du programmeur.\\ \hline
 & C’est un choix de convention de présentation du code. \\ \hline
\hline
\end{tabular}
\end{center}

\pagebreak